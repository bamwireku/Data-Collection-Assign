\documentclass[10pt,letterpaper]{article}
\begin{document}
\title{HIGH SCHOOL DROP OUTS IN EASTAFRICA}
\author{by BAMWIREKU PRISCA   \\ 216020499 \\  16/U/18716}
\maketitle
\section {Abstract }
The issue of dropout rates within minority groups has become an increasingly important problem. This is especially true for those who live in recognized at-risk communities. There is little present literature that speaks to the specific factors that affect East-African youth and their reasoning behind dropping out of school. This study attempts to analyze the differences through using interviews from knowledgeable educators and available literature/studies to present the main factors contributing to this issue and what solutions can be developed to solve it.
\section{Introduction }
Education is often purported to be the key to a successful future, and without it, attaining this success decreases dramatically. The issue of dropout rates within minority groups has become an increasingly important problem. 
 There is a lot of information regarding high school dropout rates and black youth. Unfortunately, there is no present literature that speaks directly to the issue of dropout rates and East-African youth. There are articles that speak to issues of transition and immigration regarding these youths and how this affects their livelihood in Western society. These issues are not directly related to education, but they are contributing factors to the problem posed in this paper and therefore help give this issue context. 

\section{Purpose of the Study }
This research is important and beneficial to educational administrators because it will help them have a clearer picture of this problem. More importantly, it can provide them with a basis from which they can create better programming and provide better resources for students who may be affected by this problem in the future. This research is also helpful to East-African youth and their families, where this will help increase their awareness about this issue and possibly influence positive changes to deter students from dropping out.

\section{Scope of the study}
This study seeks to find out the main causes of school dropouts among the youth of East-Africa between 8-20 years. 
\section{Methodology}
This qualitative research will include a review of the literature that relates to my topic and face-to-face interviews with the victims of school dropout. 
The quantitative part involves use of numbers to estimate the overall percentage of the dropouts. Mainly children between 12-17 with an estimated percentage of 70 percent fall victims.
.
\section{Limitations of the study}
The biggest and only limitation of this study was that I could not speak to the students that I am researching about. Being able to directly interview the youth that my research is about would have given my research a much deeper look into the issue of high school dropout rates within the East-African community. I wanted to get their own personal perspectives on why they feel that they this problem is happening, but because of the restrictions of my research I now have to solely rely on the educator’s perspective.
\section{Conclusion }
To aid this research, a collection of data from a group of people would be helpful. Sample questions that are related to this data collection survey would include;
1.The name of the victim.\\
2.The gender.\\
3The age.\\
4.The education level the victim attained.\\
5.Picture to help identify the person.\\

\end{document}